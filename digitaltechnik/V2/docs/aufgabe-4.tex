\chapter{Aufgabe 4}
\section{Aufgabe 4.2}
\paragraph{Vorteile der differentiellen Signalübertragung}
Die differentielle Signalübertragung wird in allen modernen Protokollen verwendet. Fast alle Bussysteme, die außerhalb eines Gerätes liegen, greifen auf sie zurück. Ihre Stärke liegt in einer hohen Fehlerresistenz auch bei niedrigen Spannungen, was schnelle Übertragungsraten ermöglicht. Die Übertragung eines differenziellen Signals erfolgt dazu über zwei Kabel. Während das eine Kabel positive Spannungsausschläge verwendet, überträgt das andere Kabel negative Spannungsausschläge des gleichen Betrages. Das ursprüngliche Signal wird dann durch Subtraktion der beiden einzelnen Spannungen errechnet. Der große Vorteil: Verdrillt man die beiden Kabel, so wirkt eine Störung von außen auf beide gleichermaßen. Zwar ändern sich die Spannungsausschläge, die durch die jeweiligen Kabel übertragen werden, ihre Differenz bleibt jedoch unberührt und die übermittelten Daten unbeschädigt.